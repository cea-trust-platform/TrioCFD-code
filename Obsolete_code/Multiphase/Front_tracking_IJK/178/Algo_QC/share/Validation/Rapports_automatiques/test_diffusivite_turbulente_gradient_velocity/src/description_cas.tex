Ce cas test correspond \`a la diffusion d'une gaussienne pour la temp\'erature avec une vitesse non nulle correspondant \`a un produit de sinus. Il a pour but le test du mod\`ele Gradient pour le terme sous maille $\partial_j\left(\overline{\rho} \overline{U_j/\rho} - \overline{U}_j/\overline{\rho}\right)$ de l'\'equation de conservation de la masse.

\subsection{Solution analytique}

La condition initiale sur la temp\'erature est
\begin{equation}
T\left(x,y,z\right) = 293 \left[1 + e^{-\left(\frac{x-x_0}{s_0}\right)^2-\left(\frac{y-y_0}{s_0}\right)^2-\left(\frac{z-z_0}{s_0}\right)^2}\right].
\end{equation}
La condition initiale sur la vitesse est :
\begin{align*}
V_x ={}& 2 V_0 \sin(\tau x) \sin(2 \tau z) \\
V_y ={}& 0 \\
V_z ={}& 0
\end{align*}

Soit
\begin{equation}
\pi_{j} = \overline{\rho U_j} - \overline{\rho}\overline{U}_j
\end{equation}
Le mod\`ele Gradient correspond au mod\`ele
\begin{equation}
\pi_{j} = \frac{\Delta_k^2}{12} \frac{\partial U_j}{\partial x_k} \frac{\partial \rho}{\partial x_k}
\end{equation}

On note
\begin{equation}
D_i = \frac{\partial \pi_{j}}{\partial x_j}.
\end{equation}
Dans la formulation 'Velocity', le mod\`ele est impl\'ement\'e comme
\begin{equation}
\frac{\partial \rho}{\partial t} = NS^* + D_i
\end{equation}

Notons par ailleurs, $V_0 = 0.001$ et $\tau=2\pi$.

