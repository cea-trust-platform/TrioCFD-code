Ce cas test correspond \`a la diffusion d'une temp\'erature de paroi impos\'ee \`a vitesse nulle. Il a pour but le test de l'impl\'ementation d'un diffusivit\'e turbulente dans la formulation 'Favre'.

\subsection{Solution analytique}

La condition initiale est
\begin{equation}
T\left(x,y,z\right) = 400.
\end{equation}
On impose une temp\'erature de 293 K sur la paroi basse et de 586 K sur la paroi haute.
On d\'esactive la conduction de la temp\'erature mais on active un mod\`ele sous-maille de diffusivit\'e turbulente qui, dans la formulation 'Favre' ajoute un terme diffusif \`a l'\'equation de conservation de l'\'energie.
Le mod\`ele est choisit dans le cadre de ce cas test pour avoir une diffusivit\'e turbulente constante.
Au bout d'un temps assez long, ces conditions doivent conduire \`a un profil de temp\'erature lin\'eaire dans le canal.

\subsection{V\'erification des r\'esultats de la simulation}

On compare le r\'esultat de la simulation {\textsf simu\_t} \`a la solution analytique d\'ecrite pr\'ec\'edemment {\textsf ana\_t}. La diff\'erence entre les deux est \textsf{error\_t}. Si tout se passe bien, les r\'esultats de la simulation sont proches de la solution analytique et l'erreur est tr\'es faible devant la valeur totale.


