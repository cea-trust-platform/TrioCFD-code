The Kraichnan method is based on the decomposition of a signal with Fourier coefficients.\\
We want to generate random field fluctuation $(u',v',w')$ in one point $(x,y,z) \in \partial \Omega$ of the border with correct physical properties. We know we can write the fluctuations as Fourier series. Each coefficient $l = 1, \hdots, N$ (called harmonic) depends of the following variables :
\begin{itemize}
    \item A constant $\phi^l \in \mathbb{R}$ called \textbf{phase} who are the offset between the sinusoidals.
    \item A vector $\kappa^l \in \mathbb{R}^3$ called \textbf{wave vector}. His norm is called \textbf{wave number}.
    \item A constant $A^l \in \mathbb{R}$ called the \textbf{amplitude}.
    \item An orthonal vector $\sigma^l \in \mathbb{R}^3$ to \textbf{wave vector}. It is the direction of the fluctuations.
\end{itemize}
When we have all the variables, the Fourier theory allows us to write the fluctuations field as :
\[
(u',v',w')_{(x,y,z)} = 2\sum_{l=1}^{N} A^l \cos\big(\kappa^l \cdot (x,y,z)+ \psi^l \big) \sigma^l
\]
The Kraichnan method is based on this decomposition. We generate random variables $(\phi^l, \kappa^l, A^l, \sigma^l)$ and we create the synthetic fluctuations as inlet boundary. 

