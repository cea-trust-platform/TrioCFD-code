\rhead{\'EQUATIONS DE CONSERVATIONS}
\chapter{Rappel des \'equations fondamentales de la dynamique des fluides}

Dans cette section on rappelle les \'equations fondamentales de la dynamique
des fluides. Elle permet d'introduire les principales notations et
les \'equations aux d\'eriv\'ees partielles fondamentales sans hypoth\`eses
physiques simplificatrices. Les d\'emonstrations peuvent \^etre trouv\'ees
dans les r\'ef\'erences classiques. Dans la section \ref{sub:ConservMasse},
on rappelle l'\'equation de conservation de la masse (encore appel\'ee
\'equation de continuit\'e) et dans la section \ref{sub:QDM} on rappelle
l'\'equation de conservation de la quantit\'e de mouvement. Plusieurs
formes math\'ematiques \'equivalentes entre elles existent dans la litt\'erature
: forme locale, forme locale notation vectorielle, forme locale conservative,
forme locale conservative en notation vectorielle, formes macroscopiques,
etc ... Ici on choisit la forme locale conservative avec notations
vectorielles.


\chapter{\label{sub:ConservMasse}Conservation de la masse}
\lhead{Conservation de la masse}
\rhead{\'EQUATIONS DE CONSERVATIONS}
\begin{equation}
\frac{\partial\rho}{\partial t}+\boldsymbol{\nabla}\cdot(\rho\mathbf{u})=0\label{eq:ConvMasse}
\end{equation}
o\`u $\rho\equiv\rho(\mathbf{x},\,t)$ est la densit\'e avec $\mathbf{x}$
la position et $t$ le temps, et $\mathbf{u}\equiv\mathbf{u}(\mathbf{x},\,t)$
la vitesse. D'autres formes \'equivalentes de cette \'equation peuvent
\^etre rencontr\'ees en appliquant l'identit\'e vectorielle $\boldsymbol{\nabla}\cdot(\rho\mathbf{u})=\rho\boldsymbol{\nabla}\cdot\mathbf{u}+\mathbf{u}\cdot\boldsymbol{\nabla}\rho$
et en faisant appara\^itre la d\'eriv\'ee mat\'erielle $d\rho/dt=\partial\rho/\partial t+\mathbf{u}\cdot\boldsymbol{\nabla}\rho$.


\chapter{\label{sub:QDM}Conservation de la quantit\'e de mouvement}
\lhead{Conservation de la quantit\'e de mouvement}
\rhead{\'EQUATIONS DE CONSERVATIONS}
L'\'equation de conservation de la Quantit\'e De Mouvement (QDM) traduit
le principe fondamental de la dynamique qui indique que la variation
de quantit\'e de mouvement \`a l'int\'erieur d'un volume de contr\^ole est
\'egale \`a la somme de toutes les forces ext\'erieures qui lui sont appliqu\'ees.
Les forces qui s'appliquent sur le volume \'el\'ementaires peuvent \^etre
s\'epar\'ees en forces de volume et forces de surface. Ces derni\`eres s'expriment
comme un vecteur contrainte qui agit sur une surface, et ce vecteur
contrainte s'exprime \`a son tour comme le produit scalaire d'un tenseur
des contraintes $\mathbf{T}$ (de composante $T_{ij}$) et du vecteur
normal \`a la surface $\mathbf{n}$. La contrainte totale est d\'ecompos\'ee
en deux parties : $T_{ij}=-p\delta_{ij}+\tau_{ij}$. La premi\`ere est
le tenseur des contraintes associ\'ees \`a la pression $-p\delta_{ij}$
o\`u $p\equiv p(\mathbf{x},\,t)$ est la pression et $\delta_{ij}$
est le symbole de Kronecker qui vaut un si $i=j$ et z\'ero sinon. La
seconde, not\'ee $\tau_{ij}$ est associ\'ee aux contraintes visqueuses.
La pression agit de fa\c con isotrope et sa valeur d\'epend de l'\'etat thermodynamique
du fluide. Les contraintes visqueuses sont li\'ees \`a l'\'etat de d\'eformation
du fluide

Comme pour l'\'equation de conservation de la masse, plusieurs formes
math\'ematiques et \'equivalentes entre elles peuvent \^etre d\'eduites. Son
\'ecriture sous forme locale conservative est :
\begin{equation}
\frac{\partial(\rho\mathbf{u})}{\partial t}+\boldsymbol{\nabla}\cdot(\rho\mathbf{u}\mathbf{u})=-\boldsymbol{\nabla}p+\boldsymbol{\nabla}\cdot\boldsymbol{\tau}+\mathbf{F}_{v}\label{eq:QDM}
\end{equation}


Dans l'\'equation (\ref{eq:QDM}), le membre de gauche de l'\'equation
repr\'esente la quantit\'e d'acc\'el\'eration par unit\'e de volume. Les termes
du membre de droite repr\'esentent respectivement (\emph{i}) les forces
associ\'ees \`a la pression par unit\'e de volume, (\emph{ii}) les contraintes
visqueuses par unit\'e de volume et (\emph{iii}) la force externes par
unit\'e de volume. Lorsqu'on ne consid\`ere que la gravit\'e elle s'exprime
sous la forme : $\mathbf{F}_{v}=\rho\mathbf{g}$.


\chapter{Forme du tenseur des contraintes visqueuses}
\lhead{Forme du tenseur des contraintes visqueuses}
\rhead{\'EQUATIONS DE CONSERVATIONS}
Le tenseur des contraintes visqueuses $\mathbf{\tau}$ est g\'en\'eralement
exprim\'e en fonction des taux de d\'eformation dans l'\'ecoulement. On
rappelle ci-dessous les d\'efinitions des tenseurs des taux de d\'eformation
et des taux de rotation \`a partir desquels sera exprim\'e le tenseur
des contraintes visqueuses.


\subsection*{Rappel du tenseur des taux de d\'eformation}

L'accroissement de vitesse de deux particules fluides positionn\'ees
respectivement en $\mathbf{r}$ et $\mathbf{r}+d\mathbf{r}$ et de
vitesse $\mathbf{u}$ et $\mathbf{u}+d\mathbf{u}$ s'exprime sous
la forme $du_{i}=\sum_{j=1}^{3}(\partial u_{i}/\partial x_{j})dx_{j}$
au premier ordre par rapport aux composantes $dx_{j}$ (pour $j=1,2,3$).
Dans cette expression, les quantit\'es $G_{ij}=\partial u_{i}/\partial x_{j}$
sont les \'el\'ements d'un tenseur de rang deux, le tenseur des taux de
d\'eformation du fluide (ou des gradients de vitesse). En trois dimensions,
il s'\'ecrit sous la forme d'une matrice $3\times3$ qui peut \^etre d\'ecompos\'ee
en une partie sym\'etrique et une partie antisym\'etrique :

\begin{equation}
G_{ij}=\frac{\partial u_{i}}{\partial x_{j}}=\frac{1}{2}\left(\frac{\partial u_{i}}{\partial x_{j}}+\frac{\partial u_{j}}{\partial x_{i}}\right)+\frac{1}{2}\left(\frac{\partial u_{i}}{\partial x_{j}}-\frac{\partial u_{j}}{\partial x_{i}}\right)\label{eq:TauxDeformation}
\end{equation}


Le premier terme est le tenseur des taux des d\'eformations :

\begin{equation}
S_{ij}=\frac{1}{2}\left(\frac{\partial u_{i}}{\partial x_{j}}+\frac{\partial u_{j}}{\partial x_{i}}\right)\label{eq:TenseurDeformations}
\end{equation}
et il est sym\'etrique ($S_{ij}=S_{ji}$). Le second terme est le tenseur
des taux de rotation :

\begin{equation}
\Omega_{ij}=\frac{1}{2}\left(\frac{\partial u_{i}}{\partial x_{j}}-\frac{\partial u_{j}}{\partial x_{i}}\right)\label{eq:TenseurRotations}
\end{equation}
et ce tenseur est antisym\'etrique ($\Omega_{ij}=-\Omega_{ji}$).


\subsection*{Forme du tenseur des contraintes visqueuses pour un fluide newtonien}

Lorsque les fluides sont newtoniens la relation contrainte-d\'eformation
est lin\'eaire et isotrope. La relation g\'en\'erale s'\'ecrit :

\begin{equation}
\tau_{ij}=\eta\left(2S_{ij}-\frac{2}{3}S_{kk}\delta_{ij}\right)+\zeta S_{kk}\delta_{ij}\label{eq:Contraintes-Newtonien}
\end{equation}
qui fait appara\^itre deux viscosit\'es, la viscosit\'e dynamique $\eta\equiv\eta(\mathbf{x},\,t)$
et la viscosit\'e de volume $\zeta$ (ou deuxi\`eme viscosit\'e). Le premier
terme correspond \`a une d\'eformation sans changement de volume tandis
que le second terme correspond \`a une dilatation isotrope. Dans la
majeure partie des applications on ne tient pas compte de la viscosit\'e
en volume ($\zeta=0$) et le tenseur des contraintes s'\'ecrit $\tau_{ij}=2\eta S_{ij}-(2/3)\eta S_{kk}\delta_{ij}$,
soit en utilisant la relation (\ref{eq:TenseurDeformations}) :

\begin{equation}
\tau_{ij}=\eta\left(\frac{\partial u_{i}}{\partial x_{j}}+\frac{\partial u_{j}}{\partial x_{i}}\right)-\frac{2}{3}\eta\left(\frac{\partial u_{k}}{\partial x_{k}}\right)\delta_{ij}\label{eq:Contraintes}
\end{equation}
ou encore en notations vectorielles :

\begin{equation}
\boldsymbol{\tau}=\eta(\boldsymbol{\nabla}\mathbf{u}+\boldsymbol{\nabla}^{T}\mathbf{u})-\frac{2}{3}\eta(\boldsymbol{\nabla}\cdot\mathbf{u})\mathbf{I}\label{eq:Contraintes-Deformations_Newtonien}
\end{equation}
o\`u $\mathbf{I}$ est la matrice diagonale unit\'e.


\subsection*{Fluide newtonien de viscosit\'e constante}

Lorsque la viscosit\'e du fluide $\eta$ est constante (i.e. $\eta(\mathbf{x},\,t)=\eta_{0}=\mbox{Cte}$),
le terme des contraintes visqueuses $\boldsymbol{\nabla}\cdot\boldsymbol{\tau}$
dans l'\'equation (\ref{eq:QDM}) s'exprime sous la forme :

\begin{eqnarray*}
\frac{\partial\tau_{ij}}{\partial x_{j}} & = & \eta_{0}\left[\frac{\partial^{2}u_{i}}{\partial x_{j}\partial x_{j}}+\frac{\partial}{\partial x_{j}}\left(\frac{\partial u_{j}}{\partial x_{i}}\right)\right]-\frac{2}{3}\eta_{0}\frac{\partial}{\partial x_{j}}(\boldsymbol{\nabla}\cdot\mathbf{u})\delta_{ij}\\
 & = & \eta_{0}\left[\frac{\partial^{2}u_{i}}{\partial x_{j}\partial x_{j}}+\frac{\partial}{\partial x_{i}}(\boldsymbol{\nabla}\cdot\mathbf{u})\right]-\frac{2}{3}\eta_{0}\frac{\partial}{\partial x_{i}}(\boldsymbol{\nabla}\cdot\mathbf{u})\\
 & = & \eta_{0}\boldsymbol{\nabla}^{2}u_{i}+\frac{\eta_{0}}{3}\frac{\partial}{\partial x_{i}}(\boldsymbol{\nabla}\cdot\mathbf{u})
\end{eqnarray*}


C'est-\`a-dire en notations vectorielles :

\begin{equation}
\boldsymbol{\nabla}\cdot\boldsymbol{\tau}=\eta_{0}\boldsymbol{\nabla}^{2}\mathbf{u}+\frac{\eta_{0}}{3}\boldsymbol{\nabla}(\boldsymbol{\nabla}\cdot\mathbf{u})\label{eq:ContraintVisq-constante}
\end{equation}



\chapter{R\'esum\'e}
\lhead{R\'esum\'e}
\rhead{\'EQUATIONS DE CONSERVATIONS}

\section{Cas g\'en\'eral pour un fluide newtonien}

Les \'equations de conservation de la masse et de la quantit\'e de mouvement
s'\'ecrivent sous la forme g\'en\'erale :

\begin{subequations}

\begin{eqnarray}
\frac{\partial\rho}{\partial t}+\boldsymbol{\nabla}\cdot(\rho\mathbf{u}) & = & 0\label{eq:Bilan_ConservMasse}\\
\frac{\partial(\rho\mathbf{u})}{\partial t}+\boldsymbol{\nabla}\cdot(\rho\mathbf{u}\mathbf{u}) & = & -\boldsymbol{\nabla}p+\boldsymbol{\nabla}\cdot\boldsymbol{\tau}+\rho\mathbf{F}_{v}\label{eq:Bilan_QDM}
\end{eqnarray}
o\`u $\rho\equiv\rho(\mathbf{x},\,t)$ et $\mathbf{u}\equiv\mathbf{u}(\mathbf{x},\,t)$
sont les inconnues du syst\`eme d'\'equations. La loi d'\'etat sur la pression
et l'hypoth\`ese de fluide newtonien pour le tenseur des contraintes
permettent de fermer le syst\`eme. La loi d'\'etat sur la pression et
les mod\`eles de type \og bas Mach \fg{} qui s\'eparent la pression
en une pression thermodynamique et une pression hydrodynamique sont
pr\'esent\'es dans le chapitre suivant. Pour un fluide newtonien, le tenseur
des contraintes $\mathbf{\tau}$ est reli\'e \`a celui des d\'eformations
$\mathbf{D}$ par la relation :

\end{subequations}

\begin{equation}
\boldsymbol{\tau}=\eta(\boldsymbol{\nabla}\mathbf{u}+\boldsymbol{\nabla}^{T}\mathbf{u})-\frac{2}{3}\eta(\boldsymbol{\nabla}\cdot\mathbf{u})\mathbf{I}\label{eq:Bilan_TenseurContrainte}
\end{equation}
o\`u la viscosit\'e dynamique est not\'ee $\eta\equiv\eta(\mathbf{x},\,t)$
et la viscosit\'e de volume $\zeta$ a \'et\'e n\'eglig\'ee.


\section{Cas particulier d'une viscosit\'e constante}

Lorsque la viscosit\'e est consid\'er\'ee constante (i.e. $\eta(\mathbf{x},\,t)=\eta_{0}=\mbox{Cte}$),
le terme de divergence du tenseur des contraintes $\boldsymbol{\nabla}\cdot\boldsymbol{\tau}$
se simplifie \`a l'aide de la relation (\ref{eq:ContraintVisq-constante})
et le syst\`eme d'\'equations devient :

\begin{subequations}

\begin{eqnarray}
\frac{\partial\rho}{\partial t}+\boldsymbol{\nabla}\cdot(\rho\mathbf{u}) & = & 0\label{eq:Bilan_ConservMasse-1}\\
\frac{\partial(\rho\mathbf{u})}{\partial t}+\boldsymbol{\nabla}\cdot(\rho\mathbf{u}\mathbf{u}) & = & -\boldsymbol{\nabla}p+\eta_{0}\boldsymbol{\nabla}^{2}\mathbf{u}+\frac{\eta_{0}}{3}\boldsymbol{\nabla}(\boldsymbol{\nabla}\cdot\mathbf{u})+\rho\mathbf{F}_{v}\label{eq:Bilan_QDM-1}
\end{eqnarray}


\end{subequations}


\section{Cas particulier des \'ecoulements incompressibles avec viscosit\'e variable}

Lorsque le fluide est consid\'er\'e incompressible (i.e. $\rho(\mathbf{x},\,t)=\rho_{0}=\mbox{Cte})$
alors l'\'equation de conservation de la masse devient $\boldsymbol{\nabla}\cdot\mathbf{u}=0$
(car $\partial\rho_{0}/\partial t=0$ et $\boldsymbol{\nabla}\rho_{0}=\mathbf{0}$),
le terme non lin\'eaire s'\'ecrit $\boldsymbol{\nabla}\cdot(\rho_{0}\mathbf{u}\mathbf{u})=\rho_{0}\mathbf{u}\cdot\boldsymbol{\nabla}\mathbf{u}$
et le tenseur des contraintes visqueuses (Eq. (\ref{eq:ContraintVisq-constante}))
se simplifie lui aussi en $\boldsymbol{\tau}=\eta(\boldsymbol{\nabla}\mathbf{u}+\boldsymbol{\nabla}^{T}\mathbf{u})$. 

\begin{subequations}

\begin{align}
\boldsymbol{\nabla}\cdot\mathbf{u} & =0,\label{eq:ContinuiteMP-1}\\
\rho_{0}\frac{\partial\mathbf{u}}{\partial t}+\rho_{0}\mathbf{u}\cdot\boldsymbol{\nabla}\mathbf{u} & =-\boldsymbol{\nabla}p+\boldsymbol{\nabla}\cdot\left[\eta(\boldsymbol{\nabla}\mathbf{u}+\boldsymbol{\nabla}^{T}\mathbf{u})\right]+\rho_{0}\mathbf{F}_{v}\label{eq:DBF-1-1}
\end{align}


\end{subequations}

Dans cette formulation, la viscosit\'e dynamique $\eta$ reste dans
le terme entre crochets car il peut d\'ependre de la position comme
dans les mod\`eles de turbulence.


\section{Cas particulier des \'ecoulements incompressibles avec viscosit\'e constante}

Enfin, lorsque le fluide est consid\'er\'e incompressible et de viscosit\'e
dynamique $\eta_{0}$ constante, le tenseur des contraintes visqueuses
(Eq. (\ref{eq:ContraintVisq-constante})) se simplifie une nouvelle
fois en $\boldsymbol{\nabla}\cdot\boldsymbol{\tau}=\eta_{0}\boldsymbol{\nabla}^{2}\mathbf{u}$
et le syst\`eme d'\'equations devient :

\begin{subequations}

\begin{align}
\boldsymbol{\nabla}\cdot\mathbf{u} & =0,\label{eq:ContinuiteMP}\\
\rho_{0}\frac{\partial\mathbf{u}}{\partial t}+\rho_{0}\mathbf{u}\cdot\boldsymbol{\nabla}\mathbf{u} & =-\boldsymbol{\nabla}p+\eta_{0}\boldsymbol{\nabla}^{2}\mathbf{u}+\rho_{0}\mathbf{F}_{v}\label{eq:DBF-1}
\end{align}


\end{subequations}

\newpage
\chapter{Conservation de l'\'energie}
\lhead{Conservation de l'\'energie}
\rhead{\'EQUATIONS DE CONSERVATIONS}
Plusieurs formes de l'\'equation de bilan de l'\'energie sont possibles
selon que l'on consid\`ere la conservation de l'\'energie totale, l'\'energie
interne, l'enthalpie totale ou l'enthalpie. Des formulations peuvent
\^etre d\'eduites sur la temp\'erature ou m\^eme l'entropie. Dans la suite,
on restreint la pr\'esentation \`a l'\'ecriture de la conservation de l'\'energie
interne qui sera \'ecrite sous forme \'equivalente sur l'\'equation de la
temp\'erature.

L'\'equation de bilan de l'\'energie interne $e$ s'\'ecrit \cite[p. 126]{Book_Candel}
:

\begin{equation}
\frac{\partial\rho e}{\partial t}+\boldsymbol{\nabla}\cdot(\rho e\mathbf{u})=-\boldsymbol{\nabla}\cdot\mathbf{q}-p\boldsymbol{\nabla}\cdot\mathbf{u}+\boldsymbol{\tau}:\boldsymbol{\nabla}\mathbf{u}\label{eq:Energie}
\end{equation}
Dans cette \'equation, le membre de gauche repr\'esente le taux de variation
de l'\'energie interne par unit\'e de volume. Dans le membre de droite,
le premier terme repr\'esente le flux de chaleur par unit\'e de volume
; le second terme repr\'esente la puissance des forces de pression par
unit\'e de volume ; et le dernier terme repr\'esente la puissance des
forces visqueuses par unit\'e de volume. Ce dernier est la fonction
de dissipation visqueuse qui est toujours positive ou nulle. Ainsi
les forces visqueuses entra\^inent toujours un accroissement de l'\'energie
interne du fluide et donc de sa temp\'erature.

L'\'equation de conservation de l'\'energie interne peut se reformuler
en une \'equation sur la temp\'erature $T$. Cette \'equation prend deux
formes diff\'erentes selon que l'on utilise la chaleur sp\'ecifique \`a
volume constant $C_{v}$ ou bien \`a pression constante $C_{p}$. Formul\'ee
en $C_{v}$, elle s'\'ecrit :

\begin{subequations}

\begin{equation}
\frac{\partial(\rho C_{v}T)}{\partial t}+\boldsymbol{\nabla}\cdot(\rho C_{v}T\mathbf{u})=-\boldsymbol{\nabla}\cdot\mathbf{q}-T\left(\frac{\partial p}{\partial T}\right)_{\rho}\boldsymbol{\nabla}\cdot\mathbf{u}+\boldsymbol{\tau}:\boldsymbol{\nabla}\mathbf{u}+\rho T\frac{dC_{v}}{dt}\label{eq:Energie_Cv}
\end{equation}
Formul\'ee en $C_{p}$ elle s'\'ecrit :

\begin{equation}
\frac{\partial(\rho C_{p}T)}{\partial t}+\boldsymbol{\nabla}\cdot(\rho C_{p}T\mathbf{u})=-\boldsymbol{\nabla}\cdot\mathbf{q}-\left(\frac{\partial\ln\rho}{\partial\ln T}\right)\frac{dp}{dt}+\boldsymbol{\tau}:\boldsymbol{\nabla}\mathbf{u}+\rho T\frac{dC_{p}}{dt}\label{eq:Energie_Cp}
\end{equation}

\end{subequations}

\begin{description}
\item [{Remarque}] : lorsque le gaz est consid\'er\'e parfait, i.e. $\rho=p_{th}/(RT)$
o\`u $R$ est la constante des gaz parfaits et $p_{th}$ est la pression
thermodynamique, le coefficient $(\partial\ln\rho/\partial\ln T)$
vaut $(\partial\ln\rho/\partial\ln T)=-1$. En supposant que le flux
de chaleur est d\'efini par la loi de Fourier $\mathbf{q}=-\lambda\boldsymbol{\nabla}T$
o\`u $\lambda$ est la conductivit\'e thermique, l'Eq. (\ref{eq:Energie_Cp})
devient :
\end{description}
\begin{equation}
\frac{\partial(\rho C_{p}T)}{\partial t}+\boldsymbol{\nabla}\cdot(\rho C_{p}T\mathbf{u})=\boldsymbol{\nabla}\cdot(\lambda\boldsymbol{\nabla}T)+\frac{dp_{th}}{dt}+\boldsymbol{\tau}:\boldsymbol{\nabla}\mathbf{u}+\rho T\frac{dC_{p}}{dt}\label{eq:Energie_Cp-1}
\end{equation}


L'introduction de la pression thermodynamique $p_{th}$ sera utile
pour les mod\`eles \og bas Mach \fg{}.

\newpage
\chapter{Approximation de Boussinesq en incompressible}
\lhead{Approximation de Boussinesq en incompressible}
\rhead{\'EQUATIONS DE CONSERVATIONS}
Pour des \'ecoulements incompressibles, lorsque la densit\'e est suppos\'ee
constante $\rho=\rho_{0}=\mbox{Cte}$, la densit\'e n'est ni une fonction
de la temp\'erature ni de la composition du fluide (pour les m\'elanges
miscibles). Dans ce cas, les effets de flottabilit\'e sont uniquement
pris en compte par les forces gravitationnelles. Cette simplification
est connue comme l'\og approximation de Boussinesq \fg{} et valable
en consid\'erant que la variation de densit\'e $\Delta\rho\ll\rho_{0}$.
Dans ce cas, le terme force s'\'ecrit dans les \'equations (\ref{eq:ContinuiteMP-1})--(\ref{eq:DBF-1-1})
ou (\ref{eq:ContinuiteMP})--(\ref{eq:DBF-1}) :

\begin{equation}
\mathbf{F}_{v}=-\mathbf{g}\beta_{T}(T-T_{0})\label{eq:ApproxBoussinesq}
\end{equation}
o\`u $\beta_{T}$ est le coefficient de dilatation thermique et $T_{0}$
une temp\'erature de r\'ef\'erence. Dans cette relation, le signe n\'egatif
indique que si la diff\'erence de temp\'erature est positive $\Delta T=T-T_{0}>0$
(i.e. pr\`es de la paroi chaude en convection naturelle), alors la force
est dirig\'ee dans le sens oppos\'e \`a la gravit\'e $\mathbf{g}$.


\chapter{\label{sub:Numerique-dans-TrioCFD}Num\'erique dans \texttt{TrioCFD}}
\lhead{Num\'erique dans \texttt{TrioCFD}}
\rhead{\'EQUATIONS DE CONSERVATIONS}
Dans ce document on d\'etaille les m\'ethodes num\'eriques mises en \oe uvre
dans \texttt{TrioCFD} pour le mod\`ele incompressible d\'efini par les
\'equations (\ref{eq:ContinuiteMP-1})--(\ref{eq:DBF-1-1}). Deux m\'ethodes
de discr\'etisation spatiales sont possibles dans l'outil de calculs
: la m\'ethode des Volume-El\'ements Finis (VEF) et celle des Volumes
Diff\'erences Finies (VDF) mais on ne d\'ecrit que la partie VEF. Dans
la suite, le domaine de calcul est not\'e $\Omega$.


\section{Sch\'ema en temps (valable en VEF et VDF)}

Apr\`es discr\'etisation, le syst\`eme matrice-vecteur r\'esolu s'\'ecrit :

\begin{equation}
\left\{ \begin{array}{rcl}
\delta t^{-1}\mathbf{M}U^{n+1}+\mathbf{A}U^{n+1}+\mathbf{L}(U^{n})U^{n+1}+\mathbf{B}^{T}P^{n+1} & = & F^{n}+\delta t^{-1}\mathbf{M}U^{n},\\
\mathbf{B}U^{n+1} & = & 0.
\end{array}\right.\label{eq:NavierStokes-dis-FV}
\end{equation}
o\`u $U^{n+1}\in\mathbb{R^{N_{\mathbf{u}}}}$ repr\'esente le vecteur
vitesse discr\'etis\'e au temps $(n+1)\delta t$ o\`u $\delta t$ est le
pas de temps et $N_{\mathbf{u}}$ est le nombre de degr\'es de libert\'e
pour discr\'etiser spatialement la vitesse. Les matrices en gras seront
d\'efinies ci-dessous car d\'ependantes de la discr\'etisation en espace.
$P^{n+1}\in\mathbb{R}^{N_{p}}$ repr\'esente la pression discr\'etis\'ee
au temps $(n+1)\delta t$ et $N_{p}$ est le nombre de degr\'es de libert\'e
pour discr\'etiser spatialement la pression.

Afin de d\'ecoupler la vitesse et la pression, la r\'esolution des \'equations
(\ref{eq:ContinuiteMP-1})--(\ref{eq:DBF-1-1}) est r\'ealis\'ee en trois
\'etapes \cite{Chor68,Tema68}: 
\begin{itemize}
\item \textbf{\'etape de pr\'ediction :} calculer $U^{*}$ solution de 
\[
{\delta t}^{-1}\mathbf{M}U^{*}+\mathbf{A}U^{*}+\mathbf{L}(U^{n})U^{*}+\mathbf{B}^{T}P^{n}=F^{n}+\delta t^{-1}\mathbf{M}U^{n}.
\]
\`a cette \'etape $\mathbf{B}U^{*}\neq0$. 
\item \textbf{Calcul de la pression :} calculer $P'$ solution de 
\[
\mathbf{B}\mathbf{M}^{-1}\mathbf{B}^{T}P'={\delta t}^{-1}\mathbf{B}U^{*},\quad P^{n+1}=P'+P^{n}.
\]

\item \textbf{\'etape de correction :} calculer $U^{n+1}$ solution de 
\[
\mathbf{M}U^{n+1}=\mathbf{M}U^{*}-{\delta t}\mathbf{B}^{T}P'.
\]
\end{itemize}
\begin{description}
\item [{Remarque}] : plusieurs solveurs ont \'et\'e test\'es (SIMPLE, SIMPLER,
PISO) qui, selon les probl\`emes ont montr\'e une convergence relativement
faible. Le solveur utilis\'e \`a ce jour est inspir\'e de la r\'ef\'erence \cite{Guermond-Quartapelle_IJNMF1998}.
\end{description}

\section{Sch\'ema en espace VEF }

La m\'ethode num\'erique est bas\'ee sur la m\'ethode des \'el\'ements finis de
Crouzeix-Raviart non conformes \cite{CrRa73}, et d\'etaill\'es dans \cite{Emon92,Heib03,Fort06,Angeli_etal_FVCA2017}.
Pour ($d=2$) (resp. $d=3$), on consid\`ere l'espace $xy$ de $\mathbb{R}^{2}$
(resp. l'espace $xyz$ de $\mathbb{R}^{3}$) d'origine $O$. On note
($\mathbf{e}_{x},\,\mathbf{e}_{y},\,\mathbf{e}_{z})$ ($\mathbf{e}_{1},\,\mathbf{e}_{2},\,\mathbf{e}_{3})$
les vecteurs de la base canonique. On consid\`ere $\mathcal{T}_{h}:={\displaystyle \cup}{}_{\ell=1}^{N_{T}}T_{\ell}$,
un maillage r\'egulier du domaine $\Omega$ constitu\'e de simplexes (triangles
en 2D et t\'etra\`edres en 3D), de sommets $(S_{i})_{i=1}^{N_{S}}$, o\`u
$i$ est l'indice de $N_{s}$ sommets. Le nombre de simplexes (sommets)
est not\'e par $N_{T}$ (resp. $N_{S}$). Soit $T_{\ell}\in\mathcal{T}_{h},$la
fronti\`ere de $T_{\ell}$ est constitu\'ee d'ar\^etes en 2D ou de faces
en 3D, mais on les appellera \og face \fg{} dans tous les cas. Soit
$\overline{\mathcal{F}}_{h}=\cup_{k=1}^{\overline{N}_{F}}F_{k}$ l'ensemble
des faces du maillage, et $\mathcal{F}_{h}=\cup_{k=1}^{N_{F}}F_{k}$
l'ensemble des faces int\'erieures, o\`u $\overline{N}_{F}$ (resp. $N_{F}$)
est le nombre total (resp. interne) de faces. On note $M_{k}$ le
barycentre de la face $F_{k}$, et $\mathbf{n}_{k}$ le vecteur normal
unitaire sortant \`a $F_{k}$.

Soit $P_{1}(T)$ l'ensemble des polyn\^omes d'ordre 1 d\'efinis sur $T$.
L'espace de discr\'etisation des vitesses est :

\begin{equation}
X_{h}:=\{\mathbf{v}_{h}\,|\,\forall T\in\mathcal{T}_{h},\,\mathbf{v}_{h}\in P_{1}(T)^{d}\,\mbox{et}\,\forall\,F\in\mathcal{F}{}_{h}:\,[\mathbf{v}_{h}](\mathbf{x}_{F})=\mathbf{0}\,\},\label{eq:CrRa}
\end{equation}
o\`u $\mathbf{x}_{F}$ repr\'esente le barycentre de la face $F$, et
$[\mathbf{v}_{h}](\mathbf{x}_{F})$ est le saut de $\mathbf{v}_{h}$
sur la face $F$. On suppose que $\overline{F}=\overline{T}\cap\overline{T'}$,
telle que $\mathbf{n}_{F}=\mathbf{n}_{T,F}$. Le saut $[\mathbf{v}_{h}](\mathbf{x}_{F})$
est d\'efini par : 
\[
[\mathbf{v}_{h}](\mathbf{x}_{F}):=\mathbf{v}_{h|T_{\ell}}(\mathbf{x}_{F})-\mathbf{v}_{h|T_{\ell'}}(\mathbf{x}_{F})\mbox{ if }\overline{F}=\overline{T}_{\ell}\cap\overline{T}_{\ell'},\mbox{ et }\mathbf{n}_{F}=\mathbf{n}_{T_{\ell}|F}.
\]
L'espace $X_{h}$ est muni de la semi-norme $||\mathbf{v}_{h}||_{h}=\left(\sum_{\ell=1}^{T_{\ell}}|\mathbf{v}_{h|T_{\ell}}|_{1,T_{\ell}}^{2}\right)^{1/2}$,
o\`u $|\mathbf{v}_{h|T_{\ell}}|_{1,T_{\ell}}$ est la semi-norme de
$\mathbf{v}_{h|T_{\ell}}\in H^{1}(T_{\ell})$. On note $X_{0,h}:=\{\mathbf{v}_{h}\in X_{h}\,|\,\mathbf{u}_{h|\partial\Omega}=0\}$. 

Soit $\lambda_{i}|_{T}$ la coordonn\'ees barycentrique associ\'ee au
sommet $S_{i}|_{T}$ et $\left(\mathbf{e}^{\beta}\right)_{\beta=1}^{d}$
les vecteurs de la base canonique. Les fonctions de base associ\'ees
\`a l'espace $X_{h}$ sont les vecteurs $\left(\left(\boldsymbol{\varphi}_{i}^{\beta}\right)_{i=1}^{N_{F}}\right)_{\beta=1}^{d}$
tels que $\boldsymbol{\varphi}_{i}^{\beta}|_{T}=\left(1-d\lambda_{i}|_{T}\right)\mathbf{e}^{\beta}$.
On appelle $\psi_{j}$ la fonction caract\'eristique associ\'ee au triangle
$T_{j}$.


\section{Matrices volume-\'el\'ements finis pour l'approximation P1NC/P0}

Soit $(\mathbf{u}{}_{h}^{n},\,p_{h}^{n})\in X_{h}\times L_{h}$ l'approximation
spatiale de $(\mathbf{u}^{n},\,p^{n})$ dans $X_{h}\times L_{h}$
telle que : 
\[
\mathbf{u}_{h}^{n}:=\sum_{\beta=1}^{d}\sum_{i=1}^{N_{F}}\left(U_{i}^{\beta}\right)^{n}\boldsymbol{\varphi}{}_{i}^{\beta},\quad p_{h}^{n}:=\sum_{\ell=1}^{N_{T}}P_{\ell}^{n}\psi_{\ell}.
\]
Posons : $U_{n}=(\,(U_{i}^{\beta})^{n}\,)_{\beta,i}\in\mathbb{R}^{N_{\mathbf{u}}}$,
$F^{n}=(\,(\mathbf{f}^{n+1},\boldsymbol{\varphi}_{i}^{\beta})_{0}\,)_{\beta,i}\in\mathbb{R}^{N_{\mathbf{u}}}$,
avec $N_{\mathbf{u}}:=d\,N_{F}$, et $p_{h}^{n}=(P_{\ell}^{n})_{\ell}\in\mathbb{R}^{N_{T}}$. 
\begin{itemize}
\item La matrice de masse $\mathbf{M}\in\mathbb{R}^{N_{\mathbf{u}}\times N_{\mathbf{u}}}$
est compos\'ee de $d$ blocs diagonaux \'egaux \`a $\mathbf{M}_{F}\in\mathbb{R}^{N_{F}\times N_{F}}$
tels que $\left(\mathbf{M}_{F}\right)_{i,j}=(\phi_{i},\phi_{j})_{0}$
avec $\phi_{i}|_{T}=\left(1-d\lambda_{i}|_{T}\right)$. 


Pour $d=2$, on obtient que la matrice de masse $2D$ est diagonale
: 
\[
\left(\mathbf{M}_{F}\right)_{i,j}=\delta_{ij}\sum_{\ell\,|\,M_{i}\in T_{\ell}}\frac{|T_{\ell}|}{3}.
\]
La matrice de masse 2D de la m\'ethode des \'el\'ements finis non conformes
de Crouzeix-Raviart est \'egale \`a la matrice de masse 2D obtenue par
la m\'ethode des volumes finis d\'ecrite dans la th\`ese de Emonot \cite{Emon92}.
Ce n'est plus le cas en 3D, mais dans le code \texttt{TrioCFD} c'est
la la matrice de masse de la discr\'etisation en volumes finis qui est
impl\'ement\'ee pour laquelle

\end{itemize}
\[
\left(\mathbf{M}_{F}\right)_{i,j}=\delta_{ij}\sum_{\ell\,|\,M_{i}\in T_{\ell}}\frac{|T_{\ell}|}{4}.
\]

\begin{itemize}
\item La matrice de rigidit\'e $\mathbf{A}\in\mathbb{R}^{N_{\mathbf{u}}\times N_{\mathbf{u}}}$
est compos\'ee de $d$ blocs diagonaux \'egaux \`a $\mathbf{A}_{F}\in\mathbb{R}^{N_{F}\times N_{F}}$tels
que $\left(\mathbf{A}_{F}\right)_{i,j}=(\nabla\phi_{i},\nabla\phi_{j})_{0}$.
On obtient : 
\[
\left(\mathbf{A}_{F}\right)_{i,j}=\sum_{\ell\,|\,M_{i},M_{j}\in T_{\ell}}|T_{\ell}|^{-1}\mathbf{S}_{i,\ell}\cdot\mathbf{S}_{j,\ell},
\]
o\`u $\mathbf{S}_{i,\ell}$ est le vecteur \og face normale \fg{}
associ\'e \`a la face oppos\'ee au sommet $S_{i}$ dans le triangle $T_{\ell}$.


La matrice de rigidit\'e de la m\'ethode des \'el\'ements finis non conformes
de Crouzeix-Raviart est \'egale \`a la matrice de rigidit\'e obtenue par
la m\'ethode des volumes finis d\'ecrite dans la th\`ese de Emonot.

\item La matrice de couplage $\mathbf{B}\in\mathbb{R}^{N_{T}\times N_{\mathbf{u}}}$
est compos\'ee de $d$ blocs tels que $\mathbf{B}=(\mathbf{B}^{\beta})_{\beta=1}^{d}$,
$\mathbf{B}^{\beta}\in\mathbb{R}^{N_{T}\times N_{F}}$ et :


\[
(\mathbf{B}^{\beta})_{\ell,j}=-\mathbf{S}_{j,\ell}\cdot\mathbf{e}^{\beta}
\]


\item Dans le cas o\`u la pression est $P_{1}$, on a $p_{h}^{n}=(P_{i}^{n})_{i}\in\mathbb{R}^{N_{S}}$
et la matrice de couplage $\mathbf{B}\in\mathbb{R}^{N_{S}\times N_{\mathbf{u}}}$
est telle que : 
\[
\begin{array}{rcl}
(\mathbf{B}^{\beta})_{i,k} & = & -\sum_{\ell\,|\,M_{k},S_{i}\in T_{\ell}}(\,(d+1)d\,)^{-1}\mathbf{S}_{i,\ell}\cdot\mathbf{e}^{\beta}\end{array}.
\]

\end{itemize}
