\rhead{INTRODUCTION}

\lhead{TrioCFD overview}
\chapter{TrioCFD overview}

Le code de calcul \texttt{TrioCFD} est un logiciel open-source \cite{TrioCFD}
pour les simulations en m\'ecanique des fluides, adapt\'e en particulier
aux calculs massivement parall\`eles d\textquoteright \'ecoulements turbulents
dans des configurations industrielles.\\
Il existe depuis le milieu des ann\'es 90 (et les travaux sur les Volumes-El\'ements
Finis dans \cite{Emon92}) et a trouv\'e des applications dans plusieurs
sous-domaines de la m\'ecanique des fluides.\\

Le pr\'esent document d\'ecrit les mod\`eles physiques impl\'ement\'es dans \texttt{TrioCFD}.
De nombreuses r\'ef\'erences
accessibles dans la litt\'erature (dont 28 th\`eses) d\'ecrivent les m\'ethodes
num\'eriques du code ou des applications originales. Ces r\'ef\'erences
(th\`eses et articles) peuvent \^etre retrouv\'ees au format \texttt{PDF}
sur le site internet \cite{TrioCFD}. Cette note technique compl\`ete
la description qualitative du site web en pr\'ecisant les mod\`eles math\'ematiques
qui sont utilis\'es dans les simulations. Compte tenu de la diversification
des applications depuis 1992, on restreint la description aux \'ecoulements
monophasiques newtoniens, incompressibles et turbulents. Pour ce type
d'\'ecoulements de nombreux d\'etails sont d\'ej\`a disponibles dans plusieurs
r\'ef\'erences dont \cite[etc ...]{Angeli_etal_NURETH2015,Angeli_etal_FVCA2017,TrioCFD}.
Cette note technique initie une documentation en faisant un effort
de synth\`ese de la litt\'erature existante, mais aussi d'homog\'en\'eisation
des notations aussi bien entre les \og mod\`eles physiques \fg{} et
la partie \og m\'ethodes num\'eriques \fg{} qui font souvent l'objet
de notations diff\'erentes entre les documents. Ici la partie num\'erique
n'appara\^it que dans la section \ref{sub:Numerique-dans-TrioCFD}
mais plusieurs nouvelles sections apparaitront dans les versions ult\'erieures
de la mise \`a jour de cette documentation.

Deux principales m\'ethodes de discr\'etisation peuvent \^etre utilis\'ees
dans \texttt{TrioCFD} : la m\'ethode des Volume-El\'ements Finis (VEF)
et celle des Volume-Diff\'erence Finies (VDF). Pour les VEF, les maillages
associ\'es \`a cette m\'ethode doivent n\'ecessairement \^etre triangulaires
(en 2D) ou t\'etra\'edriques (en 3D). Le code est programm\'e en langage
\texttt{C++} et est bas\'e sur la base logicielle \texttt{TRUST} (\texttt{TR}io\_\texttt{U}
\texttt{S}oftware for \texttt{T}hermalhydraulics), dans lequel sont
notamment programm\'e l\textquoteright ensemble des m\'ethodes num\'eriques
et des sch\'emas de discr\'etisation utilis\'es dans \texttt{TrioCFD}. La
base \texttt{TRUST} est aussi le noyau d'autres applications du service
STMF. Il est possible de r\'ealiser des impl\'ementations locales dans
le code via le concept de \texttt{BALTIK} (\texttt{B}uilding an \texttt{A}pplication
\texttt{L}inked to \texttt{T}r\texttt{I}o\_U \texttt{K}ernel), qui
correspond \`a une brique de code (un ensemble de fichiers \texttt{.cpp}
et \texttt{.h}) \`a modifier, puis de les int\'egrer \'eventuellement dans
le code source une fois les d\'eveloppements v\'erifi\'es et valid\'es. Pour
r\'esumer, \texttt{TrioCFD} s'appuie sur \texttt{TRUST} mais cette documentation
ne concerne que la partie \texttt{TrioCFD}.



\lhead{Document organisation}
\chapter{Document organisation}

Afin de pr\'eciser les mod\`eles math\'ematiques ainsi que leur domaine
de validit\'e, il est n\'ecessaire de faire un rappel des \'equations de
conservation de masse, de quantit\'e de mouvement et d'\'energie sans
consid\'eration d'hypoth\`ese simplificatrice. C'est l'objet de la section
\ref{sec:Equations-de-conservations} dans laquelle les \'equations
seront pr\'esent\'ees sous leur forme la plus g\'en\'erale. Dans la mesure
du possible, les mod\`eles math\'ematiques sont \'ecrits aussi bien en notations
vectorielles qu'en notations indicielles. En notation vectorielle,
les vecteurs, les tenseurs et les matrices sont \'ecrits en gras. Dans
cette section, des rappels sont \'egalement effectu\'es sur les d\'efinitions
des tenseurs du taux de d\'eformation et celui de rotation car ceux-ci
sont utilis\'es dans la section suivante d\'edi\'ee aux mod\`eles de turbulence.
Les mod\`eles de turbulence en m\'ecanique des fluides sont nombreux et
plusieurs livres et publications de synth\`ese existent d\'ej\`a
(e.g. \cite{Book_Chassaing,Argyropoulos-Markatos_ReviewTurbulence_AMM2015}).
Dans \texttt{TrioCFD}, les mod\`eles qui sont d\'evelopp\'es sont ceux qui
sont les plus couramment utilis\'es dans la litt\'erature et qui font
l'objet d'un consensus dans le domaine tels que la LES (Smagorinski
et WALE) et les diff\'erentes d\'eclinaisons du RANS $\overline{k}-\overline{\epsilon}$.
Ils seront d\'etaill\'es dans la section \ref{sec:Modeles-de-turbulence}.\\

Une base de cas tests d'environ 200 rapports de validations existe pour \texttt{TrioCFD}.
La liste des cas tests de cette base est pr\'esent\'e en Annexe A
du \emph{Rapport de validation} (voir TrioCFD\_validatio\_report.pdf).
Dans les versions ult\'erieures de cette documentation, les mod\`eles d\'ecrits seront
illustr\'es par quelques rapports de validation.
Plusieurs mises \`a jour de cette documentation sont \`a pr\'evoir pour l'\'elargir
petit \`a petit \`a l'ensemble des mod\`eles accessibles dans le code.
