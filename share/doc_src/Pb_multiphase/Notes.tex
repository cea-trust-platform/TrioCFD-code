\documentclass[a4paper,12pt]{article}

\usepackage{conf}

\title{Documentation CMFD}
\author{Alan \textsc{Burlot} \and Clément \textsc{Bazin} \and Andrew \textsc{Peitavy} \and Corentin \textsc{Reiss} \and Antoine \textsc{Gerschenfeld}}

\date{\today}

\begin{document}

\maketitle

\section{Introduction}

\section{Software architecture}
Dans cette section, on décrit la structure informatique de Pb\_Multiphase. On parle du jdd, de la construction des correlations, de la structure des équations. On reprend la présentation d'Antoine qui décrit comment est codé Pb\_Multiphase et PolyMAC

\section{Models descriptions}
On reprend les modèles décrits dans le document déjà rédigé par Corentin. On l'étoffe. Idéalement, il faudrait flécher chaque modèle sur un cas-test.

\section{Best practices}
\subsection{How to add a physical model?}
The steps to follow if one wants to add a new lift force model or a new something: where to go in the sources, how to manage matrices
\subsection{How to handle discretization for sources?}
The steps to follow
\subsection{How to add a turbulence model?}
The steps to follow
\subsection{How to create a new field and postprocess it?}
The steps to follow


%%%%
- remonter matrix filling et analytical terms as source terms dans architecture
- coef de frottement dans Mixture modeling
- chaleur et injection masse avant diamètre
- corriger les * par \times
- vérification : signe =, si possible \text{} autour des mots clefs trust/trio, mettre des \parent{} ou \croc{} 
- mot clef trust en \texttt{}
- physical modeling -> bifluid phys mod
- ajouter des phrases/paragraphes en début de chapitre/section pour dire ce qu'il y a dans le chapitre/section


\end{document}
