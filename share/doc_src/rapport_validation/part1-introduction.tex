The TrioCFD database contains currently around 160 test cases of validation which are called in this document "validation sheets".
This present report is the result of an inventory work of all those validation sheets to know (non exhaustive list): What type of flow they simulate ? What is the degree of maturity of each validation ? and so on ... \smallskip\newline
Moreover, until now, the whole validation sheets are placed in various folders of TrioCFD package, hence some of them cannot be found quickly. This report gathers in one single document fifteen validation sheets which were classified into five parts according to the type of flow they model.\smallskip\newline
The five parts are successively presented for flows of increasing complexity modeling:\smallskip \newline
$\Rightarrow$ Laminar Flow - Part III\smallskip \newline
$\Rightarrow$ Thermal Laminar Flow - Part IV\smallskip\newline
$\Rightarrow$ Turbulent Flow - Part V\smallskip\newline
$\Rightarrow$ Thermal Turbulent Flow - Part VI\smallskip\newline
$\Rightarrow$ Two-phase Flows - Part VII\smallskip \newline
The ten sheets that appear in this document are selected because they separately simulate a particular flow that is well-known in the CFD literature such as the "lid-driven cavity flow" (Laminar Flow) or the "Vahl Davis convection flow" (Thermal Laminar Flow). Other sheets were selected because they compare TrioCFD with other CFD results using alternative academic or commercial codes (e.g. Fluent or other benchmarks) such as "OBI diffuser" (Turbulent flow). When available and representative of flows, experimental data measured on facility tests were included for comparisons and appear on the graphs. The number of sheets in this report will be gradually increased for the future versions of TrioCFD.\smallskip\newline
In older versions of TrioCFD, each validation sheet has been written by different authors who used their own structure. In order to improve the readability of this report, the structure of validation sheets has been harmonized by using identical titles of sections (and same number) for all sheets. For that purpose, a new PRM template has been updated in TrioCFD (v1.8.2) as described in Part II. That new template will be helpful for future versions of this document when a specific study using TrioCFD will lead to write a new PRM sheet or update an old one. \newline

Finally, this validation report has several objectives:\smallskip\newline
- Allow users to identify fields of applications of the TrioCFD code;\smallskip\newline
- Give users examples of modeling (TrioCFD keywords and boundary conditions) on specific cases;\smallskip\newline
- Inform users of changes/improvements on code validation;\smallskip\newline
- Make an inventory of the code validation for each version delivered with the physical and/or numerical impacts observed following developments, corrections or modification of test cases that have been made between each version;\smallskip\newline
- Update the validation status of the code.
