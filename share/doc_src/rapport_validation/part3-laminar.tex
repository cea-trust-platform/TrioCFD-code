%\baselineskip=16pt
\vspace*{1.5cm}
In this first part of the document, the test cases with laminar flows are considered. Let us remind that a flow is considered as laminar when the Reynolds number is lower than (approximately) 2000. The Reynolds
number (Re) is a dimensionless parameter that is defined as the ratio between the inertial forces over the viscous forces ($Re=\rho U_{0}/\nu$, where $\rho$ is the fluid density, $\nu$ is the kinematic viscosity and $U_{0}$
a characteristic velocity of the system). When the Reynolds number is greater than 2000, the flow is considered as turbulent. The validation cases of turbulent flows will be considered in Part V of this document. In what
follows, three academic cases are detailed:\vspace*{0.5cm}\newline
\hspace*{0.5cm} $\bullet$ Poiseuille flow\vspace*{0.5cm}\newline
\hspace*{0.5cm} $\bullet$ Lid driven cavity flow\vspace*{0.5cm}\newline
\hspace*{0.5cm} $\bullet$ Cylinder in laminar Cross Flow for Re = 100\vspace*{3cm}\newline
\begin{center}\includegraphics[width=7cm]{tools/jet_laminaire.jpg}\end{center}
